\setacronymstyle{long-short}

\newglossaryentry{transpilation}{
	name={transpilation},
	description={Act of compiling high-level language code to high-level code of another language. This term is mostly used in context of JavaScript and its subsidary languages building on top of the language.}
}
\newglossaryentry{antlr4}{
	name={Antlr4},
	description={The fourth version of the ANTLR project, which enables the generation of a lexer, parser and related tools through the use of a grammar file that defines the language's structure. See~\cite{antlr} for more information.}
}
\newglossaryentry{abstract-syntax-tree}{
	name={abstract syntax tree},
	description={A hierarchical, tree-like representation of the abstract syntactic structure of source code. Each node corresponds to a construct in the code, such as expressions or statements, providing a simplified view of the code's logical structure. Essential in compilers for tasks such as semantic analysis and output generation.}
}
\newacronym{ast}{AST}{\Gls{abstract-syntax-tree}}
\newglossaryentry{gnu-compiler-collection}{
	name={GNU compiler collection},
	description={GCC is an open-source compiler system that supports languages like C, C++, and Fortran. It converts source code into machine code or \gls{intermediate-representation}, enabling program execution across different platforms. It is widely used in software development for its efficiency and portability.}
}
\newacronym{gcc}{GCC}{\Gls{gnu-compiler-collection}}
\newglossaryentry{intermediate-representation}{
	name={intermediate representation},
	description={In compiler design, the Intermediate Representation (IR) is an abstract, machine-independent code form used between the source and target code. It simplifies compilation, supports optimizations, and enables portability across architectures. IR can be linear (e.g., SSA) or tree-like (e.g., AST) and is central to modern compiler pipelines.}
}
\newacronym{ir}{IR}{\Gls{intermediate-representation}}
\newglossaryentry{backus–naur-form}{
	name={Backus–Naur Form},
	description={Backus-Naur form is a }
}
\newacronym{bnf}{BNF}{\Gls{backus–naur-form}}
\newglossaryentry{object-oriented-programming}{
	name={Object Oriented Programming},
	description={}
}
\newacronym{oop}{OOP}{\Gls{object-oriented-programming}}

% Usage:
% \gls{label} lowercase in text
% \Gls{label} Uppercase in text
% \newacronym{label}{abbrev}{full}
% \newglossaryentry{label}{settings}
