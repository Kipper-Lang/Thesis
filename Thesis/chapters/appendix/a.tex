Kipper was not originally conceived as a thesis project or a high-level language in its current form but began as a minor personal project around September 2021. The language evolved over time due to an interest in building upon the JavaScript ecosystem while addressing its commonly perceived limitations. Initially, Kipper incorporated fundamental features such as arithmetic operations, low-level data types, functions, and built-in functions. With the introduction of major features—including error recovery, complex object types, runtime types, and classes—the language has significantly increased in complexity. In its core functionality, Kipper can now be compared to early versions of languages such as JavaScript or Python. This thesis represents a substantial advancement for what was originally a small-scale project. With continued development, Kipper is expected to achieve a full core feature set and direct integration with existing environments in the coming years. However, as of the time of writing, while Kipper is feature-complete within its planned scope, it does not yet include all features commonly found or required in modern programming languages. Essential functionalities such as imports, modules, and asynchronous operations are currently absent from the language.

Furthermore, this thesis should be regarded as a snapshot of the project's development rather than a comprehensive account of its entire history. It primarily focuses on the key features implemented during the 2024/25 period, when Kipper took its current form. Additionally, this thesis does not assert that the language will strictly adhere to all specifications outlined in this paper, nor does it guarantee backward compatibility with the described features. As Kipper remains in active development, it will continue evolving to achieve a standard feature set suitable for a major v1.0 release.

%%% Local Variables:
%%% mode: LaTeX
%%% TeX-master: "../thesis"
%%% End: