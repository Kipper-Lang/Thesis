\begin{spacing}{1}
    \chapter*{Abstract}
\end{spacing}
\begin{wrapfigure}{r}{0.4\textwidth}
    \begin{center}
      \includegraphics[height=0.4\textwidth]{pics/Kipper-Logo.png}
    \end{center}
\end{wrapfigure}

This thesis examines the challenges present in JavaScript environments, particularly the lack of runtime types and comprehensive type safety, and introduces Kipper as a solution. Kipper is a high-level programming language designed to enforce strict type safety at runtime, addressing limitations found in JavaScript and TypeScript, where type correctness is either unchecked or dependent on static analysis tools.

To establish the foundation for Kipper's design, this thesis analyses the JavaScript ecosystem, identifying key issues related to type handling and security. Additionally, potential technologies for implementing Kipper are evaluated, considering their suitability for achieving the project's objectives. Based on this groundwork, the Kipper type system is introduced, incorporating runtime type determination, pattern matching, and interface-based duck typing to enforce type correctness dynamically.

A key focus of this thesis is the development of the Kipper compiler, which translates Kipper code into JavaScript or TypeScript while preserving its type safety mechanisms. The implementation details, including architectural choices and optimizations, are discussed in depth. By addressing the shortcomings of JavaScript and TypeScript, Kipper provides a structured and predictable approach to safer web development. Future work will focus on extending the language's feature set and refining its integration with existing technologies.

\begin{spacing}{1}
	\chapter*{Zusammenfassung}
\end{spacing}
\begin{wrapfigure}{r}{200px}
	\begin{center}
		\includegraphics[height=200px]{pics/Kipper-Logo.png}
	\end{center}
\end{wrapfigure}

Diese Arbeit untersucht die Herausforderungen in JavaScript-Umgebungen, insbesondere das Fehlen von Laufzeittypen und umfassender Typsicherheit, und stellt Kipper als Lösung vor. Kipper ist eine High-Level-Programmiersprache, die entwickelt wurde, um strenge Typsicherheit zur Laufzeit zu erzwingen. Sie behandelt die Einschränkungen, die in JavaScript und TypeScript zu finden sind, wo die Typkorrektheit entweder ungeprüft oder abhängig von statischen Analysetools ist.

Um die Grundlage für das Design von Kipper zu schaffen, wird in dieser Arbeit das JavaScript-Ökosystem analysiert und die wichtigsten Probleme im Zusammenhang mit der Typbehandlung und Sicherheit identifiziert. Darüber hinaus werden potentielle Technologien für die Implementierung von Kipper auf ihre Eignung zur Erreichung der Projektziele hin untersucht. Auf dieser Grundlage wird das Kipper-Typsystem vorgestellt, das Laufzeit-Typbestimmung, Pattern-Matching und Interface-basiertes Ducktyping zur dynamischen Durchsetzung von Typkorrektheit beinhaltet.

Ein Schwerpunkt dieser Arbeit ist die Entwicklung des Kipper-Compilers, der Kipper-Code in JavaScript oder TypeScript unter Beibehaltung der Typsicherheitsmechanismen übersetzt. Die Implementierungsdetails, einschließlich Architekturentscheidungen und Optimierungen, werden abgewogen. Durch die Behebung der strukturellen Probleme von JavaScript und TypeScript bietet Kipper einen konsistenten und konsequenten Ansatz für eine sicherere Webentwicklung. In der Zukunft wird sich das Projekt auf die Erweiterung des Funktionsumfangs der Sprache und die Verfeinerung der Integration mit bestehenden Technologien konzentrieren.

%%% Local Variables:
%%% mode: LaTeX
%%% TeX-master: "../thesis"
%%% End: