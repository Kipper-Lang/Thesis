\section{Potential Features}
\label{sec:future}
\setauthor{Lorenz Holzbauer}

Currently, Kipper primarily serves web developers by offering integration with the TypeScript (TS) and JavaScript (JS) ecosystems. This focus aligns with the widespread use of JavaScript on both client and server sides. However, the web development landscape is evolving, and there is a growing demand among developers for high-performance solutions.

\subsection{WebAssembly Support}
%https://webassembly.org/
%https://wasmer.io/

WebAssembly is an open standard that enables high-performance, portable code execution across diverse environments, including browsers, servers, and embedded systems. Adding WebAssembly as a compilation target for Kipper provides a potential avenue for enhancing performance and allows Kipper to function as a standalone server application. Emerging technologies such as Wasmer demonstrate that WebAssembly could also support cloud-based container environments while occupying only a fraction of the size of traditional containers.

Implementing WebAssembly as a target for Kipper presents several challenges. Unlike TypeScript and JavaScript, which are dynamically typed and compatible with Kipper’s runtime model, WebAssembly requires a more structured type system and memory management approach. This necessitates that the compiler handle low-level memory management, including garbage collection. Additionally, a compatibility layer with a runtime would need to be developed to manage Kipper’s dynamic data requirements.

Given these challenges, it is unlikely that WebAssembly support will be included in Kipper in the near future.

\subsection{IDE Support}

IDE support is an important factor in fostering developer adoption and productivity. Providing tooling within popular IDEs can position Kipper as a viable option for developers working on TypeScript (TS) and JavaScript (JS) projects. Effective IDE integration can contribute to code quality and the overall development process.

A key requirement for Kipper IDE support is syntax highlighting. Highlighting keywords, data types, and structures specific to Kipper can improve code readability and reduce cognitive load for developers. This feature is available in the web editor of Kipper. IntelliSense and autocompletion are also significant components, enabling IDEs to provide context-aware suggestions for code completion, parameter hints, and documentation pop-ups, similar to the functionality common in TS and JS environments.

Static analysis tools can assist in maintaining code quality. Integrating a Kipper-specific linter within IDEs allows developers to receive feedback on potential errors, best practices, and stylistic consistency during development.

Initial development efforts may focus on Visual Studio Code due to its widespread use and extensibility. However, consideration for additional IDEs, such as JetBrains products including WebStorm and IntelliJ IDEA, can help accommodate different developer preferences and workflows. Broader IDE compatibility may increase accessibility and support diverse development environments.

\section{Integration with other languages}
\setauthor{Luna Klatzer}

\section{Project Result}
\setauthor{Luna Klatzer}

%%% Local Variables:
%%% mode: LaTeX
%%% TeX-master: "../thesis"
%%% End:
