\section{Future}
\label{sec:future}
\setauthor{Lorenz Holzbauer}

Currently, Kipper caters primarily to web developers by providing seamless integration with the TS and JS ecosystems. This focus aligns well with the popularity of JavaScript on both the client and server sides. However, the web development landscape is rapidly evolving, and developers are increasingly seeking high-performance solutions.

\subsection{WebAssembly Support}
%https://webassembly.org/
%https://wasmer.io/

WebAssembly is an open standard that enables high-performance, portable code execution across diverse environments, including browsers, servers, and embedded systems. Adding WebAssembly as a compilation target for Kipper presents a compelling opportunity to enhance performance. In addition it enables Kipper to run on as a standalone server application. With new technologies like wasmer, WebAssembly could power containers in the cloud while being a fraction of the size of a traditional container.

Implementing WebAssembly as a target for Kipper is not without challenges. Unlike TS/JS, which are dynamically typed and inherently compatible with Kipper’s runtime model, Wasm requires a more rigorous type system and memory management model. This means the compiler needs to handle low-level memory management with garbage collection. Additionally, it would be neccessary to implement a compatibility layer with a runtime to handle Kipper's dynamic data requirements.

Due to this issues, it is unlikely that WebAssembly support will land in Kipper in the next few versions.

\subsection{IDE Support}
IDE support plays a crucial role in increasing developer adoption and productivity. By providing advanced tooling within popular IDEs, Kipper can establish itself as a practical choice for developers working on TS and JS projects. Enhanced IDE integration contributes to code quality and the overall developer experience.

One of the foundational requirements for Kipper IDE support is robust syntax highlighting. Highlighting keywords, data types, and structures unique to Kipper ensures code readability and reduces cognitive load for developers. This already implemented in the web-editor of Kipper. IntelliSense and autocompletion are equally important. IDEs can offer context-aware suggestions for code completion, parameter hints, and documentation pop-ups, mirroring the functionality developers expect in TS and JS environments.

Static analysis tools are important for maintaining code quality. By integrating a Kipper-specific linter within IDEs, developers can receive real-time feedback on potential errors, best practices, and stylistic consistency.

While initial efforts may focus on Visual Studio Code due to its popularity and extensibility, other IDEs like the JetBrains products "Webstorm" and "IntelliJ Idea" should also be considered. Broad IDE support ensures that Kipper is accessible to a wider audience, accommodating developers’ preferences.

%%% Local Variables:
%%% mode: LaTeX
%%% TeX-master: "../thesis"
%%% End:
