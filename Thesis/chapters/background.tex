\section{Dissecting the current issues}
\setauthor{Luna Klatzer}

\subsection{The JavaScript problem}

JavaScript, originally developed by Netscape in 1995 to enable interactive web pages, has become the foundational programming language for modern web browsers with 60\% of developers using the language in their profession as of 2023~\cite{jetbrains2023}. While its initial scope was limited to enhancing the functionality of websites, JavaScript has since evolved into a versatile language that serves as the foundation for diverse applications, including those outside traditional browser environments. This success has been largely made possible by its versatile and modifiable nature allowing the language to take many shapes with a relatively consistent and easy-to-use syntax.

However, this rapid expansion was not accompanied by fundamental changes to the language's initial design, leading to inherent limitations and challenges when addressing complex, large-scale systems. In modern web development, JavaScript's role extends far beyond front-end scripting. Its omnipresence is reflected in its adoption across back-end platforms (e.g. Node.js), desktop applications (e.g. Electron), and mobile development frameworks (e.g. React Native). Accompanying this growth is a vast ecosystem of libraries, frameworks, and tools that offer developers flexibility in solving specific challenges. Despite these advantages, the language presents significant difficulties for developers.

\subsubsection{Type Ambiguity and Uncontrolled Flexibility}

JavaScript’s dynamic typing provides developers with flexibility and expressiveness in their code. However, this flexibility can also lead to ambiguity and unintended behavior. Unlike statically typed languages, JavaScript does not enforce type constraints, meaning variable types are determined at runtime. It permits almost any value to be assigned to any variable without restrictions, relying solely on the developer to manage type consistency. This lack of enforcement makes the language prone to errors caused by implicit type coercion or unexpected values. Additionally, JavaScript does not inherently handle type errors and assumes that most operations are valid, even providing special cases for operations that would typically be considered invalid.

For instance:

\begin{itemize}
	\item{
		A variable intended to store a number can unexpectedly hold a string due to developer oversight or API misuse, as can be seen in listing~\ref{lst:background:uncheckedvariabletypes}.
		
		\begin{lstlisting}[language=JavaScript,caption=Unchecked variable assignments due to missing type definitions,label=lst:background:uncheckedvariabletypes]
let id; // Variable allowing any value

...

let resp = resp.json(); // Object: { id: "1234", ... }
id = resp.id; // Assigns a string even though that was not intended
		\end{lstlisting}
	}
	\item{
		Implicit conversions, such as treating `null` or `undefined` as valid inputs in arithmetic operations, often yield confusing results, as can be seen in listing~\ref{lst:background:problematicflexibility}.

		\begin{lstlisting}[language=JavaScript,caption=Broad ability to perform "invalid" operations despite clear error case,label=lst:background:problematicflexibility]
let discountRate = cart.appliedDiscount; // Let's assume it's actually "applied_discount" not "appliedDiscount" so it returns 'undefined'

return price * (1 - discountRate); // -> yields NaN (Not A Number)
		\end{lstlisting}
	}
\end{itemize}

Such flexibility complicates debugging, as issues may only surface during runtime. This increases the risk of critical bugs making it into production and requires developers to rely heavily on additional tooling or rigorous testing to mitigate the risks inherent in dynamic typing.

\subsubsection{Runtime-Bound Error Handling \& Catching}

Error handling in JavaScript is largely runtime-bound, except for syntax errors, and relies on tools like try-catch blocks and asynchronous error handling with Promise chains. This is because JavaScript is an interpreted language, meaning the program does not perform checks before executing each line. As a result, developers lack pre-execution validation.

While try-catch blocks can handle most errors, typically those defined by the developer, JavaScript allows certain operations that would be illegal in other languages, like accessing non-existent properties by simply returning undefined instead of an error. This can make the underlying cause of an issue difficult to identify, with the code potentially failing at a different location, sometimes long after the original problem has occurred.

Examples include:

\begin{itemize}
    \item{
		Accessing a missing property, which returns `undefined`, and later receiving an error due to `undefined` not being an object, as demonstrated in listing~\ref{lst:background:missingpropertyerror}.
		
		\begin{lstlisting}[language=JavaScript,caption=Accessing a missing property returns undefined which later causes an error,label=lst:background:missingpropertyerror]
// Let's assume this is some kind of API response that incorrectly returned some data
let user = {
	name: "Alice"
};

...

const userAddress =  user.address; // Undefined property ('address' does not exist), returns simply 'undefined'

...

let shippingCost = getShippingCost(address.city); // -> TypeError: Cannot read property 'city' of undefined
		\end{lstlisting}
	}
	\item{
		Misaligned function arguments, such as passing an object where a string was expected, going unnoticed until execution, as illustrated in listing~\ref{lst:background:misalignedargs}.
		
		\begin{lstlisting}[language=JavaScript,caption=Misaligned function arguments going unnoticed,label=lst:background:misalignedargs]
function greet(name) {
	console.log("Hello, " + name + "!");
}

...

let user = { firstName: "Alice" };

greet(user); // Unintended behavior: "Hello, [object Object]!"
		\end{lstlisting}
	}
\end{itemize}

\subsubsection{Modern approaches}

These limitations force a defensive programming approach, requiring the developer to anticipate and safeguard against potential failures. While third-party tools such as linters or test frameworks can help identify issues, they are generally not equivalent to built-in, language-level type and error guarantees.

Given though that JavaScript is so prominent and can hardly be replaced in the modern web and server-side space, alternative solutions have been developed in response to these and other challenges to improve JavaScript's reliability and ease of use. One of the most prominent and widely adopted of these is TypeScript, which like the topic of this thesis, implements a \gls{transpilation}-based language with independent libraries and functionality building on top of the existing JavaScript environment.

\subsection{TypeScript - One of many solutions}

TypeScript has emerged as the most widely adopted enhancement to JavaScript, functioning as a statically typed superset of the language. It introduces features such as object-oriented programming constructs and compile-time type checking, aligning its capabilities with those of traditionally typed languages like Java or C\#. By providing type annotations and a robust compilation process, TypeScript enables developers to build safer applications compared to their JavaScript counterparts. Errors related to type mismatches, for instance, can be identified during development, reducing the likelihood of runtime failures and improving overall code reliability.

Despite its advantages, TypeScript is constrained by its core design philosophy of maintaining full compatibility with JavaScript. This approach allows developers to easily integrate TypeScript with existing JavaScript codebases, promoting incremental adoption. However, it also imposes limitations on the language’s capabilities. For example, because JavaScript was not originally designed with type safety in mind, the TypeScript compiler operates as a static analysis tool, enforcing type rules only at compile time. This design choice ensures compatibility but leaves runtime type enforcement unaddressed. Consequently, developers must rely on a "trust-based" system, wherein the correctness of types is assumed during runtime based on the accuracy of their compile-time annotations.

These constraints highlight the challenges inherent in adapting a dynamically typed language to support static typing. While TypeScript significantly mitigates many of JavaScript’s shortcomings, its reliance on compile-time type checking alone limits its ability to provide comprehensive runtime guarantees, requiring developers to remain vigilant when integrating with dynamically typed JavaScript components.

\subsubsection{Unchecked compile-time casts}

As already mentioned TypeScript works on a compile-time-only basis, which does not allow for any runtime type checks. That also naturally means any standard functionality like casts can also not be checked for, since such type functionality requires the language to be able to reflect on its type structure during runtime. Given the fact though that casts, which allow the developer to narrow the type of a value down, are a necessity in everyday programs, TypeScript is forced to provide what you can call "trust-based casts". The developer can, like in any other language, specify what a specific value is expected to be, but unlike usual casts are primarily unchecked, meaning you can, if you want, cast anything to anything with no determined constraint.

While in principle this maintains the status quo and provides the developer with more freedom, it also opens up another challenge that must be looked out for when writing code. If one of those casts goes wrong and isn't valid, the developer will only know that at runtime and will have no assistance to fix it. To overcome this developers can themselves implement runtime type checks, which prevent type mismatches in ambiguous contexts. While it is a common approach, it is fairly impractical and adds a heavy burden on the developer as it requires constant maintenance and recurring rewrites to ensure the type checks are up-to-date and valid.

Let's look at an example~\ref{lst:background:uncheckedcompilecasts}.

\begin{lstlisting}[language=TypeScript,caption=Unchecked compile-time casts in TypeScript,label=lst:background:uncheckedcompilecasts]
class SuperClass {
	name: string = "Super class";
}

class MiddleClass extends SuperClass {
	superField: SuperClass = new SuperClass();
	
	constructor() {
		super();
	}
}

class LowerClass extends MiddleClass {
	classField: MiddleClass = new MiddleClass();
	
	constructor() {
		super();
	}
}

const c1 = <MiddleClass>new SuperClass(); // Unchecked cast
console.log(c1.superField.name); // Runtime Error! Doesn't actually exist

const c2 = <LowerClass>new MiddleClass(); // Unchecked cast
console.log(c2.classField.superField.name); // Runtime Error! Doesn't actually exist
\end{lstlisting}

Here we have a simple example of an inheritance structure, where we access the properties of a child that is itself also another object. Due to the nature of TypeScript operations such as casts are mostly unchecked and usually work on the base of trusting the developer to know what they're doing. That means that in the example given above, the compiler does not realise that the operation the developer is performing is invalid and will result in a failure at runtime (can't access property "name", c1.superField is undefined). Furthermore, given that JavaScript only reports on such errors when a property on an undefined value is accessed, the undefined variable may go unused for a while before it is the cause of any problem. This leads to volatile code that can in many cases not be guaranteed to work unless the developer actively pays attention to such errors and makes sure that their code does not unintentionally force unchecked casts or other similar untyped operations.

\subsubsection{Ambiguous dynamic data}

Another similar issue occurs when dealing with dynamic or untyped data, which does not report on its structure and as such is handled as if it were a JavaScript value, where all type checks and security measures are disabled. This for one makes sense given the goal of ensuring compatibility with the underlying language, but it also creates another major problem where errors regarding any-typed values can completely go undetected. Consequently, if we were to receive data from a client or server we can not ensure that the data we received is fully valid or corresponds to the expected pattern. This is a problem that does not have a workaround or a solution in TypeScript.

For example~\ref{lst:background:ambigdata}:

\begin{lstlisting}[language=TypeScript,caption=Ambiguous dynamic data in TypeScript,label=lst:background:ambigdata]
interface Data {
	x: number;
	y: string;
	z: {
		z1: boolean;
	}
}

function receiveUserReq(): object {
	// ...
	return {
		x: "1",
		y: "2",
		z: true
	}
}

var data = <Data>receiveUserReq(); // Unsafe casting with unknown data
console.log(data.z.z1); // No Runtime Error! But returns "undefined"
\end{lstlisting}

For the most part, developers are expected to simply watch out for such cases and implement their own security measures. There are potential libraries which can be utilised to add runtime checks which check the data received, but such solutions require an entirely new layer of abstraction which must be managed manually by a developer. This additional boilerplate code also increases the complexity of a program and has to be actively maintained to keep working. 

Good examples of technologies that provide runtime object schema matching are "Zod"~\cite{zod} and "joi"~\cite{joi}. Both are fairly popular and actively used by API developers who need to develop secure endpoints and ensure accurate request data. While they are a good approach to fixing the problem after the fact, they still create their own difficulties. We will examine these later in the implementation section, where we will more thoroughly compare Kipper's approach to other tools.

\section{How could it have been better}
\setauthor{Fabian Baitura}

\subsection{Case study: Java}

TypeScript has emerged as the most widely adopted enhancement to JavaScript, functioning as a statically typed superset of the language. It introduces features such as object-oriented programming constructs and compile-time type checking, aligning its capabilities with those of traditionally typed languages like Java or C\#. By providing type annotations and a robust compilation process, TypeScript enables developers to build type-safe applications. Errors related to type mismatches, for instance, can be identified during development, reducing the likelihood of runtime failures and improving overall code reliability. Despite its advantages, TypeScript is constrained by its core design philosophy of maintaining full compatibility with JavaScript. This approach allows developers to seamlessly integrate TypeScript with existing JavaScript codebases, promoting incremental adoption. However, it also imposes limitations on the language’s capabilities. For example, because JavaScript was not originally designed with type safety in mind, the TypeScript compiler operates as a static analysis tool, enforcing type rules only at compile time. This design choice ensures compatibility but leaves runtime type enforcement unaddressed. Consequently, developers must rely on a "trust-based" system, wherein the correctness of types is assumed during runtime based on the accuracy of their compile-time annotations. These constraints highlight the challenges inherent in adapting a dynamically typed language to support static typing. While TypeScript significantly mitigates many of JavaScript’s shortcomings, its reliance on compile-time type checking alone limits its ability to provide comprehensive runtime guarantees, requiring developers to remain vigilant when integrating with dynamically typed JavaScript components.

\subsection{Case study: Rust}

Rust is a systems programming language designed to offer memory safety without a garbage collector. One of the standout features of Rust is its ownership system, which enforces strict rules for memory allocation and deallocation, preventing common bugs like null pointer dereferencing or data races in concurrent programming. This guarantees memory safety at compile-time without needing a runtime environment to manage memory, unlike languages such as Java and JavaScript, which use garbage collection to manage memory dynamically.

Rust's type system is strongly and statically typed, like Java, but it emphasizes immutability and borrowing concepts to manage data lifetimes and concurrency safely. Unlike Java, Rust does not have reflection, but it provides powerful meta-programming features via macros. Rust also promotes zero-cost abstractions, ensuring that high-level abstractions have no runtime overhead, making it a popular choice for applications requiring both performance and safety. Despite working on the basis of a completely different programming paradigm it still manages to be type-safe or more accurately memory-safe. The compiler makes sure that there are no ambiguities left that could potentially lead to runtime errors and provides absolute safety in a way that still allows a certain freedom to the developer.

\subsection{Drawing comparisons to JavaScript}

Unlike the two languages we've just described, JavaScript is rather unique in its design and structure. As already mentioned, there is no proper reflection system, enforced type checks or type safety when running code, only really throwing errors when there is no other way around it. Moreover, you can say that JavaScript has no design or structure at all, and was more conceptualised as a fully dynamic type-less language with no OOP support in mind. This has caused quite a few problems in the years following the original version of JavaScript, as it has more and more developed into an OOP language while not providing any proper type functionality commonly present in such systems. Even languages like Python, which is also a dynamic interpreted language, provide static type hints and checks to ensure proper type safety when writing code.

Nonetheless, as JavaScript is currently one of the most important languages out there, the system can under no circumstances be changed as it would break backwards compatibility with previous systems and destroy the web as we know it today. This has caused quite a dilemma, which persists until today. Many tools like TypeScript have been developed since then and are seen as the de-facto solution for these problems, but it's a rather bad solution given all the current restraints, unavoidable edge cases and vulnerabilities that can be easily introduced.

\section{Tackling the issue at its core}
\setauthor{Lorenz Holzbauer}

As we have already mentioned, JavaScript is a language that can under no circumstances be changed or it would mean that most websites would break in newer browser versions. This phenomenon is also often described as "Don't break the web", the idea that any new functionality must incorporate all the previous standards and systems to ensure that older websites work and look the same. Naturally this also then extends to TypeScript, which has at its core a standard JavaScript system that can also not be changed or altered to go against the ECMAScript standard. Consequently, the most effective approach to ensuring a safe development environment for developers is to build upon JavaScript by extending its standard functionalities through a custom, unofficial system that incorporates the necessary structures and safety measures.

This is where Kipper comes into play—a language that implements a custom system, which is later transpiled into JavaScript or TypeScript. By incorporating an additional runtime and non-standard syntax, Kipper enables runtime type checks, addressing gaps left by TypeScript's compile-time-only system. This approach is particularly advantageous, as it allows the system to extend beyond JavaScript standards, introducing structures tailored to the requirements of modern programs. Accordingly, developers can rely on Kipper to enhance code security and ensure that no dynamic structures remain unchecked or bypass the type system due to edge cases.

