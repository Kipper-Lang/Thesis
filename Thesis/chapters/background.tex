\section{Dissecting the current issues}
\setauthor{Luna Klatzer}

\subsection{The JavaScript problem}

JavaScript, originally developed by Netscape in 1995 to enable interactive web pages, has become the foundational programming language for modern web browsers with 60\% of developers using the language in their profession as of 2023~\cite{jetbrains2023}. While its initial scope was limited to enhancing the functionality of websites, JavaScript has since evolved into a versatile language that serves as the foundation for diverse applications, including those outside traditional browser environments. This success has been largely made possible by its versatile and modifiable nature allowing the language to take many shapes with a relatively consistent and easy-to-use syntax.

However, this rapid expansion was not accompanied by fundamental changes to the language's initial design, leading to inherent limitations and challenges when addressing complex, large-scale systems. In modern web development, JavaScript's role extends far beyond front-end scripting. Its omnipresence is reflected in its adoption across back-end platforms (e.g. Node.js), desktop applications (e.g. Electron), and mobile development frameworks (e.g. React Native). Accompanying this growth is a vast ecosystem of libraries, frameworks, and tools that offer developers flexibility in solving specific challenges. Despite these advantages, the language presents significant difficulties for developers.

\subsubsection{Type Ambiguity and Uncontrolled Flexibility}

JavaScript’s dynamic typing provides developers with flexibility and expressiveness in their code. However, this flexibility can also lead to ambiguity and unintended behaviour. Unlike statically typed languages, JavaScript does not enforce type constraints, meaning variable types are determined at runtime. It permits almost any value to be assigned to any variable without restrictions, relying solely on the developer to manage type consistency. This lack of enforcement makes the language prone to errors caused by implicit type coercion or unexpected values. Additionally, JavaScript does not inherently handle type errors and assumes that most operations are valid, even providing special cases for operations that would typically be considered invalid.

For instance:

\begin{itemize}
	\item{
		A variable intended to store a number can unexpectedly hold a string due to developer oversight or API misuse, as can be seen in listing~\ref{lst:background:uncheckedvariabletypes}.

		\begin{lstlisting}[language=JavaScript,caption=Unchecked variable assignments due to missing type definitions,label=lst:background:uncheckedvariabletypes]
let id; // Variable allowing any value

...

let resp = resp.json(); // Object: { id: "1234", ... }
id = resp.id; // Assigns a string even though that was not intended
		\end{lstlisting}
	}
	\item{
		Implicit conversions, such as treating \lstinline|null| or \lstinline|undefined| as valid inputs in arithmetic operations, often yield confusing results, as can be seen in listing~\ref{lst:background:problematicflexibility}.

		\begin{lstlisting}[language=JavaScript,caption=Broad ability to perform "invalid" operations despite clear error case,label=lst:background:problematicflexibility]
let discountRate = cart.appliedDiscount; // Assuming it is actually "applied_discount" not "appliedDiscount" so it returns "undefined"

return price * (1 - discountRate); // -> yields NaN (Not A Number)
		\end{lstlisting}
	}
\end{itemize}

Such flexibility complicates testing and error correction, as issues may only surface during runtime. This increases the risk of critical bugs making it into production and requires developers to rely heavily on additional tooling or rigorous testing to mitigate the risks inherent in dynamic typing.

\subsubsection{Runtime-Bound Error Handling \& Catching}

Error handling in JavaScript is largely runtime-bound, except for syntax errors, and relies on tools like try-catch blocks and asynchronous error handling with Promise chains. This is because JavaScript is an interpreted language, meaning the program does not perform checks before executing each line. As a result, developers lack pre-execution validation.

While try-catch blocks can handle most errors, typically those defined by the developer, JavaScript allows certain operations that would be illegal in other languages, like accessing non-existent properties by simply returning undefined instead of an error. This can make the underlying cause of an issue difficult to identify, with the code potentially failing at a different location, sometimes long after the original problem has occurred.

Examples include:

\begin{itemize}
    \item{
		Accessing a missing property, which returns "undefined", and later receiving an error due to "undefined" not being an object, as demonstrated in listing~\ref{lst:background:missingpropertyerror}.

		\begin{lstlisting}[language=JavaScript,caption=Accessing a missing property returns undefined which later causes an error,label=lst:background:missingpropertyerror]
// Assuming this is some kind of API response that incorrectly returned some data
let user = {
	name: "Alice"
};

...

const userAddress =  user.address; // Undefined property ("address" does not exist), returns simply "undefined"

...

let shippingCost = getShippingCost(address.city); // -> TypeError: Cannot read property "city" of undefined
		\end{lstlisting}
	}
	\item{
		Misaligned function arguments, such as passing an object where a string was expected, going unnoticed until execution, as illustrated in listing~\ref{lst:background:misalignedargs}.

		\begin{lstlisting}[language=JavaScript,caption=Misaligned function arguments going unnoticed,label=lst:background:misalignedargs]
function greet(name) {
	console.log("Hello, " + name + "!");
}

...

let user = { firstName: "Alice" };

greet(user); // Unintended behavior: "Hello, [object Object]!"
		\end{lstlisting}
	}
\end{itemize}

\subsubsection{Modern approaches}

These limitations force a defensive programming approach, requiring the developer to anticipate and safeguard against potential failures. While third-party tools such as linters or test frameworks can help identify issues, they are generally not equivalent to built-in, language-level type and error guarantees.

Given though that JavaScript is so prominent and can hardly be replaced in the modern web and server-side space, alternative solutions have been developed in response to these and other challenges to improve JavaScript's reliability and ease of use. One of the most prominent and widely adopted of these is TypeScript, which like the topic of this thesis, implements a \gls{transpilation}-based language with independent libraries and functionality building on top of the existing JavaScript environment.

\subsection{TypeScript - One of many solutions}

TypeScript has emerged as the most widely adopted enhancement to JavaScript, functioning as a statically typed super-set of the language. It introduces features such as object-oriented programming constructs and compile-time type checking, aligning its capabilities with those of traditionally typed languages like Java or C\#. By providing type annotations and a robust compilation process, TypeScript enables developers to build safer applications compared to their JavaScript counterparts. Errors related to type mismatches, for instance, can be identified during development, reducing the likelihood of runtime failures and improving overall code reliability.

Despite its advantages, TypeScript is constrained by its core design philosophy of maintaining full compatibility with JavaScript. This approach allows developers to easily integrate TypeScript with existing JavaScript code bases, promoting incremental adoption. However, it also imposes limitations on the language’s capabilities. For example, because JavaScript was not originally designed with type safety in mind, the TypeScript compiler operates as a static analysis tool, enforcing type rules only at compile time. This design choice ensures compatibility but leaves runtime type enforcement unaddressed. Consequently, developers must rely on a "trust-based" system, wherein the correctness of types is assumed during runtime based on the accuracy of their compile-time annotations.

These constraints highlight the challenges inherent in adapting a dynamically typed language to support static typing. While TypeScript significantly mitigates many of JavaScript’s shortcomings, its reliance on compile-time type checking alone limits its ability to provide comprehensive runtime guarantees, requiring developers to remain vigilant when integrating with dynamically typed JavaScript components.

\subsubsection{Unchecked compile-time casts}

As already mentioned TypeScript works on a compile-time-only basis, which does not allow for any runtime type checks. That also naturally means any standard functionality like casts can also not be checked for, since such type functionality requires the language to be able to reflect on its type structure during runtime. Given the fact though that casts, which allow the developer to narrow the type of a value down, are a necessity in everyday programs, TypeScript is forced to provide what can be called "trust-based casts". This means, that TypeScript trusts the developer to check for type compatibility. The language therefore allows any type to be cast to any other type.

While in principle this maintains the status quo and provides the developer with more freedom, it also opens up another challenge that must be looked out for when writing code. If one of those casts goes wrong and is not valid, the developer will only know that at runtime and will have no assistance to fix it. To overcome this developers can themselves implement runtime type checks, which prevent type mismatches in ambiguous contexts. While it is a common approach, it is fairly impractical and adds a heavy burden on the developer as it requires constant maintenance and recurring rewrites to ensure the type checks are up-to-date and valid.

\begin{lstlisting}[language=TypeScript,caption=Unchecked compile-time casts in TypeScript,label=lst:background:uncheckedcompilecasts]
class SuperClass {
	name: string = "Super class";
}

class MiddleClass extends SuperClass {
	superField: SuperClass = new SuperClass();

	constructor() {
		super();
	}
}

class LowerClass extends MiddleClass {
	classField: MiddleClass = new MiddleClass();

	constructor() {
		super();
	}
}

const c1 = <MiddleClass>new SuperClass(); // Unchecked cast
console.log(c1.superField.name); // Runtime Error! "c1.superField" does not actually exist

const c2 = <LowerClass>new MiddleClass(); // Unchecked cast
console.log(c2.classField.superField.name); // Runtime Error! "c2.classField.superField" does not actually exist
\end{lstlisting}

In listing~\ref{lst:background:uncheckedcompilecasts}, we have a simple example of an inheritance structure, where we access the properties of a child that is itself also another object. Due to the nature of TypeScript operations such as casts are mostly unchecked and usually work on the base of trusting the developer to know what they are doing. That means that in the example given above, the compiler does not realise that the operation the developer is performing is invalid and will result in a failure at runtime (can not access property "name", c1.superField is undefined). Furthermore, given that JavaScript only reports on such errors when a property of an undefined value is accessed, the undefined variable may go unused for a while before it is the cause of any problem. This leads to volatile code that can in many cases not be guaranteed to work unless the developer actively pays attention to such errors and makes sure that their code does not unintentionally force unchecked casts or other similar untyped operations.

\subsubsection{Ambiguous dynamic data}

A similar issue occurs when dealing with dynamic or untyped data, which does not report on its structure and as such is handled as if it were a JavaScript value, where all type checks and security measures are disabled. This for one makes sense given the goal of ensuring compatibility with the underlying language, but it also creates another major problem where errors regarding any-typed values can completely go undetected. Consequently, if we were to receive data from a client or server we can not ensure that the data we received is fully valid or corresponds to the expected pattern. This is a problem that does not have a workaround or a solution in TypeScript.

An example of such a situation is given in listing~\ref{lst:background:ambigdata}:

\begin{lstlisting}[language=TypeScript,caption=Ambiguous dynamic data in TypeScript,label=lst:background:ambigdata]
interface Data {
	x: number;
	y: string;
	z: {
		z1: boolean;
	}
}

function receiveUserReq(): object {
	// ...
	return {
		x: "1",
		y: "2",
		z: true
	}
}

var data = <Data>receiveUserReq(); // Unsafe casting with unknown data
console.log(data.z.z1); // No Runtime Error! But returns "undefined"
\end{lstlisting}

For the most part, developers are expected to simply watch out for such cases and implement their own security measures. There are potential libraries which can be utilised to add runtime checks, but such solutions require an entirely new layer of abstraction which must be managed manually by a developer. This additional boilerplate code also increases the complexity of a program and has to be actively maintained to keep working.

Good examples of technologies that provide runtime object schema matching are "Zod"~\cite{zod} and "joi"~\cite{joi}. Both are fairly popular and actively used by API developers who need to develop secure endpoints and ensure accurate request data. While they are a good approach to fixing the problem, they still create their own difficulties. We will examine these later in the implementation section, where we will more thoroughly compare Kipper's approach to other tools.

\section{Examples of runtime type checking in other languages}
\setauthor{Lorenz Holzbauer}

\subsection{Case study: Java}

Java is a statically typed object-oriented programming language that runs on the Java Virtual Machine (JVM). It is widely used in enterprise applications, Android development, and large-scale systems due to its robust type system and memory management capabilities. Unlike JavaScript and TypeScript, which Kipper \gls{transpile}s to, Java enforces type safety at both compile-time and runtime. This enforcement ensures that type errors are caught early and that unexpected behaviour due to type inconsistencies is minimized.

One of Java’s key runtime type features is its use of 	reflection, which allows a program to inspect and modify its own structure during execution. Reflection enables Java to perform dynamic type checking, instantiate objects of arbitrary classes, and call methods dynamically based on type information. This mechanism is crucial for frameworks such as Spring and Hibernate, which rely on runtime type inspection for dependency injection and object-relational mapping.

\subsubsection{Runtime Type Checking}

Java enforces strict runtime type checking to ensure the correctness of operations involving polymorphism and casting. The JVM performs type checks before executing certain operations, preventing unsafe conversions. An example of this is listing~\ref{lst:background:runtimetypechecksjava}. In this example, the~\lstinline|instanceof| operator ensures that the object is of type~\lstinline|Dog| before performing a downcast. This prevents a~\lstinline|ClassCastException|, which would otherwise occur if the object were of an incompatible type.


\begin{lstlisting}[language=Java,caption=Runtime typechecks in Java,label=lst:background:runtimetypechecksjava]
class Animal {
	void makeSound() {
		System.out.println("Some sound");
	}
}

class Dog extends Animal {
	void makeSound() {
		System.out.println("Bark");
	}
}

public class TypeCheckExample {
	public static void main(String[] args) {
		Animal myAnimal = new Dog(); // Upcasting
		if (myAnimal instanceof Dog) {
			Dog myDog = (Dog) myAnimal; // Safe downcasting
			myDog.makeSound();
		}
	}
}
\end{lstlisting}

\subsubsection{Reflection}

Java provides a rich set of reflection APIs that allow the program to examine and manipulate class structures at runtime. The~\lstinline|java.lang.reflect| package enables dynamic method invocation, field modification, and even dynamic class loading. In listing~\ref{lst:background:javareflection}, the Java reflection API allows to retrieve method names and invoke them dynamically. This is particularly useful in frameworks and tools that require runtime type information, such as dependency injection frameworks and dynamic proxies.

\begin{lstlisting}[language=Java,caption=Reflection in Java,label=lst:background:javareflection]
import java.lang.reflect.Method;

class ExampleClass {
	public void sayHello() {
		System.out.println("Hello, world!");
	}
}

public class ReflectionExample {
	public static void main(String[] args) throws Exception {
		Class<?> c = ExampleClass.class;
		Object obj = c.getDeclaredConstructor().newInstance();
		
		for (Method method : c.getDeclaredMethods()) {
			System.out.println("Method: " + method.getName());
			method.invoke(obj);
		}
	}
}
\end{lstlisting}

Java’s runtime type system offers strong guarantees by enforcing type safety through reflection, dynamic type checking, and explicit exception handling. In contrast, JavaScript’s dynamic nature makes it more flexible but also prone to type-related errors.

\subsection{Case study: Rust}

Rust is a statically typed systems programming language designed for performance and safety. Unlike Java, which relies on a garbage collector and runtime type checks, Rust enforces strict compile-time guarantees to ensure memory safety and type correctness. This results in minimal runtime overhead, making Rust an appealing choice for high-performance applications. However, Rust does provide mechanisms for runtime type inspection when necessary, such as trait objects and dynamic dispatch.

Rust’s type system is centered around ownership, borrowing, and lifetimes, ensuring memory safety at compile time without the need for a garbage collector. Despite its strong static typing, Rust allows for limited runtime type checking through the Any trait and dynamic trait objects. Unlike JavaScript, Rust minimizes dynamic type handling in favor of compile-time guarantees.

\subsubsection{Runtime Type Checking}

Rust does not depend on runtime type checking as Java or JavaScript do. Instead, it enforces type safety at compile time. Nevertheless, Rust provides runtime tools for handling dynamically typed values, notably the~\lstinline|Any| trait from the~\lstinline|std::any| module. This trait enables type inspection and downcasting at runtime, functioning similarly to Java's~\lstinline|instanceof| operator.

In listing~\ref{lst:background:runtimetypecheckrust}, the~\lstinline|Any| trait is employed to examine and downcast a trait object to a specific concrete type. The~\lstinline|is::()| method checks whether a value belongs to a certain type, while~\lstinline|downcast_ref::()| provides a safe way to obtain a reference to the underlying type.


\begin{lstlisting}[language=Rust,caption=Runtime type checking in Rust,label=lst:background:runtimetypecheckrust]
use std::any::Any;

fn check_type(value: &dyn Any) {
	if value.is::<i32>() {
		println!("Value is an i32!");
	} else if value.is::<String>() {
		println!("Value is a String!");
	} else {
		println!("Unknown type");
	}
}

fn main() {
	let num = 42;
	let text = "Hello".to_string();
	
	check_type(&num);
	check_type(&text);
}
\end{lstlisting}

While the~\lstinline|Any| trait enables some level of runtime type checking, its use is generally discouraged in idiomatic Rust. The language prefers static polymorphism through generics and trait bounds, ensuring type correctness at compile time.

\subsubsection{Trait Objects and Dynamic Dispatch}

Rust’s trait system allows for polymorphism without runtime type checking. However, when dynamic dispatch is required, Rust provides trait objects, which enable runtime method resolution similar to Java’s virtual method table (vtable) approach. Unlike Java’s reflection-based system, Rust’s trait objects still enforce strict type constraints at compile time while allowing method calls to be dynamically resolved at runtime.

In listing~\ref{lst:background:rusttraitobjects}, a trait~\lstinline|Animal| is defined, and trait objects (~\lstinline|&dyn Animal|) are used to allow polymorphism. This enables different implementations to be stored in the same collection and accessed dynamically.

\begin{lstlisting}[language=Rust,caption=Trait objects and dynamic dispatch in Rust,label=lst:background:rusttraitobjects]
trait Animal {
	fn make_sound(&self);
}

struct Dog;
impl Animal for Dog {
	fn make_sound(&self) {
		println!("Bark");
	}
}

struct Cat;
impl Animal for Cat {
	fn make_sound(&self) {
		println!("Meow");
	}
}

fn main() {
	let animals: Vec<&dyn Animal> = vec![&Dog, &Cat];
	for animal in animals {
		animal.make_sound(); // Dynamically dispatched
	}
}
\end{lstlisting}

This mechanism provides dynamic polymorphism without exposing the entire type system to runtime modification as in Java’s reflection API. Unlike Java’s reflection, Rust does not allow modifying class structures at runtime, maintaining stricter safety guarantees.

Rust’s runtime type system differs significantly from Java’s and JavaScript’s approaches. Java relies on runtime reflection and dynamic type checking to handle polymorphism and introspection, whereas Rust enforces stricter compile-time type safety and only allows limited runtime type inspection through~\lstinline|Any| and dynamic trait objects. JavaScript, on the other hand, is dynamically typed, meaning type errors may only surface at runtime, whereas Rust eliminates most type-related errors before execution.


\section{Tackling the issue at its core}
\setauthor{Lorenz Holzbauer}

As we have already mentioned, JavaScript is a language that can under no circumstances be changed or it would mean that most websites would break in newer browser versions. This phenomenon is also often described as "Do not break the web", the idea that any new functionality must incorporate all the previous standards and systems to ensure that older websites work and look the same. Naturally this also then extends to TypeScript, which has at its core a standard JavaScript system that can also not be changed or altered to go against the ECMAScript standard. Consequently, the most effective approach to ensuring a safe development environment for developers is to build upon JavaScript by extending its standard functionalities through a custom, unofficial system that incorporates the necessary structures and safety measures.

The above approach is exactly the aim of Kipper—a language that implements a custom runtime type system, which is later \gls{transpile}d into JavaScript or TypeScript. By incorporating an additional runtime and non-standard syntax, Kipper enables runtime type checks. This approach is particularly advantageous, as it allows the system to extend beyond JavaScript standards, introducing structures tailored to the requirements of modern programs. Accordingly, developers can rely on Kipper to enhance code security and ensure that no dynamic structures remain unchecked or bypass the type system due to edge cases.

%%% Local Variables:
%%% mode: LaTeX
%%% TeX-master: "../thesis"
%%% End: