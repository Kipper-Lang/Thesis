\setauthor{Lorenz Holzbauer}

This chapter explores various sample use cases for Kipper to showcase its capabilities and demonstrate the environments in which it can be utilized. To keep things concise and focused, the examples provided will cover simple scenarios. More complex applications are beyond the scope of this paper, but can be explored in further documentation (see \nameref{chapter:appendix_b} for further details).

\section{Example Program}
\setauthor{Lorenz Holzbauer}

Listing \ref{lst:demo:isprime} demonstrates the computation of the prime factors of a given number using Kipper. The program is given a number, determines the prime factors and prints the result.

\begin{lstlisting}[language=Kipper,caption=A basic programs that determines the prime factors of an integer, label=lst:demo:isprime]
def primeFactors(n: num) -> void {
	while (n % 2 == 0) {
		print(2);
		n /= 2;
	}
	
	for (var i: num = 3; i * i <= n; i += 2) {
		while (n % i == 0) {
			print(i);
			n /= i;
		}
	}
	
	if (n > 2) {
		print(n);
	}
}

primeFactors(18);
\end{lstlisting}

\section{Working example using CLI}
\setauthor{Lorenz Holzbauer}

Kipper provides a command-line interface (CLI) that allows users to compile and execute programs directly from the terminal. To run the above example using the CLI, the following steps are required:

\begin{enumerate}
	\item Install the Kipper CLI globally using npm: \lstinline|npm install -g @kipper/cli|
	\item Save the Kipper source code in a file, e.g., \texttt{prime.kip}.
	\item Compile and execute a Kipper file: \lstinline|kipper run prime.kip|
\end{enumerate}

After execution, the prime factors of the given number are printed to the console.

\section{Working example in the web}
\setauthor{Lorenz Holzbauer}

Kipper can also be run in the browser using the \texttt{@kipper/web} package. Listing \ref{lst:demo:kipperbrowser} demonstrates how to include Kipper in an HTML file and execute the same prime factorization program.

\begin{lstlisting}[language=HTML,caption=Running Kipper in the browser, label=lst:demo:kipperbrowser]
<!DOCTYPE html>
<head>
	<title>Kipper Demo</title>
	<script src="https://cdn.jsdelivr.net/npm/@kipper/web@latest
	/kipper-standalone.min.js"></script>
</head>
<body>
	<script type="module">
		const logger = new Kipper.KipperLogger((level, msg) => {
			console.log(`[${Kipper.getLogLevelString(level)}] ${msg}`);
		});
		
		const compiler = new Kipper.KipperCompiler(logger, {});
		
		const code = ''; // Replace with prime.kip
		
		compiler.compile(
			code, 
			{ target: new KipperJS.TargetJS() }
		).then((result) => {
			const jsCode = result.write();
			eval(jsCode);
		});
	</script>
</body>
\end{lstlisting}

When opened in a browser, this file will execute the compiled JavaScript version of the prime factorization program, displaying the results in the browser console.

\section{Working example using Node.js}
\setauthor{Lorenz Holzbauer}



Kipper can also be utilized within a Node.js environment. The following steps outline how to run a Kipper program using Node.js:

\begin{enumerate}
	\item Install the Kipper runtime package via npm: \lstinline|npm install @kipper/core|
	\item Save the Kipper source code in a file, e.g., \texttt{prime.kip}.
	\item Use a simple Node.js script to compile and execute the Kipper code.
\end{enumerate}

Listing \ref{lst:demo:kippernode} demonstrates how to execute Kipper code within a Node.js environment.

\begin{lstlisting}[language=JavaScript,caption=Executing Kipper code in Node.js, label=lst:demo:kippernode]
import { promises as fs } from "fs";
import * as kipper from "@kipper/core";
import * as kipperJS from "@kipper/target-js";

const path = "prime.kip";
fs.readFile(path, "utf8" as BufferEncoding).then(async (fileContent: string) => {

	const logger = new kipper.KipperLogger((level, msg) => {
		console.log(`[${level}] ${msg}`);
	});
	const compiler = new kipper.KipperCompiler(logger);
	
	let result = await compiler.compile(fileContent, {

		target: new kipperJS.TargetJS(),
	});
	let jsCode = result.write();
	
	eval(jsCode);
});
\end{lstlisting}

Executing this script will transpile the Kipper code to JavaScript and run it within Node.js, printing the prime factors to the console.

