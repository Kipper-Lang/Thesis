\setauthor{Lorenz Holzbauer}

\section{Compiler API}
\label{sec:compiler_api}

The Kipper Compiler API provides the necessary functionality to compile Kipper source code into JavaScript or TypeScript. The main entry points into the API are the \lstinline|KipperCompiler| class and the \lstinline|KipperProgramContext| class.

\subsection{Initializing the Compiler}
\label{subsec:compiler_init}

To compile Kipper code, an instance of \lstinline|KipperCompiler| must be created. The compiler requires a logger instance and an optional configuration object.

\begin{lstlisting}[language=Typescript, caption=Initializing the Kipper Compiler, label=lst:compiler_initialization]
const logger = new Kipper.KipperLogger((level, msg) => {
	console.log(`[${Kipper.getLogLevelString(level)}] ${msg}`);
});

const compiler = new Kipper.KipperCompiler(logger, {});
\end{lstlisting}

The logger instance provides error reporting functionality, making it an essential part of the compilation process.

\subsection{Compiling Kipper Code}
\label{subsec:compiling}

Once the compiler is initialized, it can be used to transpile Kipper code into JavaScript or TypeScript. The \lstinline|compile| method takes the source code as a string and a configuration object specifying the compilation target.

\begin{lstlisting}[language=Typescript, caption=Compiling Kipper Code to JavaScript, label=lst:compile_example]
const result = await compiler.compile(
	`print("Hello world!");`,
	{ target: new KipperJS.TargetJS() }
);
const jsCode = result.write();

// Execute the compiled JavaScript
eval(jsCode);
\end{lstlisting}

The compilation result is an object containing the generated JavaScript or TypeScript code, which can be executed or written to a file.

\subsection{Compilation Options}
\label{subsec:compilation_options}

The \lstinline|compile| method accepts a configuration object that allows customization of the compilation process. The most important options are:

\begin{itemize}
	\item \lstinline|target|: Specifies the output language. Available targets are \lstinline|TargetJS| for JavaScript and \lstinline|TargetTS| for TypeScript.
	\item \lstinline|filename|: Specifies the filename of the generated code
	\item \lstinline|optimisationOptions|: Currently
	\lstinline|optimiseInternals| and \lstinline|optimiseBuiltIns| are available. They can be enabled by setting them to \lstinline|true|
\end{itemize}

\section{Target API}
\label{sec:target_api}

The Kipper compiler supports multiple compilation targets, allowing the transpilation of Kipper code into different languages. Developers can extend Kipper by implementing custom targets.

\subsection{Using Predefined Targets}
\label{subsec:using_targets}

Kipper provides predefined targets for JavaScript and TypeScript. These can be used as follows:

\begin{lstlisting}[language=Typescript, caption=Using Compilation Targets, label=lst:using_targets]
const jsTarget = new KipperJS.TargetJS();
const tsTarget = new KipperJS.TargetTS();
\end{lstlisting}

\subsection{Creating Custom Targets}
\label{subsec:custom_targets}

To create a custom target, a new class must extend \lstinline|KipperTarget| and implement the \lstinline|transpile| method.

\begin{lstlisting}[language=Typescript, caption=Creating a Custom Compilation Target, label=lst:custom_target]
class CustomTarget extends Kipper.KipperTarget {
	transpile(ast) {
		// Custom transpilation logic
		return "// Transpiled code";
	}
}
\end{lstlisting}

This allows for custom language backends beyond JavaScript and TypeScript.

\section{CLI Interface}
\label{sec:cli_interface}

The Kipper CLI provides a command-line interface for compiling Kipper code. The CLI is available via the \lstinline|@kipper/cli| package.

\subsection{Installing the CLI}
\label{subsec:cli_installation}

To install the Kipper CLI, use the following command:

\begin{lstlisting}[language=bash, caption=Installing Kipper CLI, label=lst:cli_install]
npm install @kipper/cli
\end{lstlisting}

This installs Kipper in the current NPM project. To install Kipper globally, use the \lstinline|-g| flag. This may require superuser privileges, depending on the installation location.

\subsection{Compiling a File}
\label{subsec:cli_compile}

To compile a Kipper source file, use the \lstinline|kipper compile| command:

\begin{lstlisting}[language=bash, caption=Compiling a Kipper File, label=lst:cli_compile]
kipper compile source.kip --target js
\end{lstlisting}

This command transpiles \lstinline|source.kip| to JavaScript. It is possible to specify the option \lstinline|--target=ts| to switch to the TypeScript target. When no target option is specified, the default target is JavaScript.

\subsection{Running a Kipper Program}

To run a Kipper source file, use the \lstinline|kipper run| command:

\begin{lstlisting}[language=bash, caption=Running a Kipper File, label=lst:cli_run]
	kipper run source.kip
\end{lstlisting}

This command transpiles to the specified target and then executes the code.

\subsection{Creating a New Project}
\label{subsec:cli_new_project}

To create a new Kipper project, use:

\begin{lstlisting}[language=bash, caption=Creating a New Kipper Project, label=lst:cli_new_project]
kipper new my_project
\end{lstlisting}

This generates a new project with the necessary configuration files.

%%% Local Variables:
%%% mode: LaTeX
%%% TeX-master: "../thesis"
%%% End: